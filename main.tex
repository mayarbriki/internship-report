\documentclass[12pt,a4paper]{report}

\usepackage{graphicx} % Ajouté pour l'inclusion des images




\begin{document}




% ----------------------
% PAGE DE GARDE
% ----------------------
\begin{titlepage}
    \centering
    \vspace*{2cm}
    {\Huge \textbf{Rapport de Stage} \par}
    \vspace{1.5cm}
    {\LARGE Développement d’un logiciel du chiffrage\par}
    \vspace{0.5cm}
    {\Large \textbf{eco-pilot} \par}
    \vspace{2cm}
    \includegraphics[width=0.4\textwidth]{logo.png}\par % Mets ton logo ici
    \vfill
    \begin{flushright}
        \textbf{Stagiaire :} Mayar Briki \\
        \textbf{Encadrant académique :} Nom Prénom \\
        \textbf{Encadrant professionnel :} Nom Prénom \\
        \textbf{Date :} \today
    \end{flushright}
\end{titlepage}

% ----------------------
% REMERCIEMENTS
% ----------------------
\chapter*{Remerciements}
Je tiens à remercier \dots

\tableofcontents
\clearpage

% ----------------------
% INTRODUCTION
% ----------------------
\chapter{Introduction}
Présenter le contexte du stage, la problématique, les objectifs, et une brève description de l’application \textbf{eco-pilote}.  

% ----------------------
% ENTREPRISE
% ----------------------
\chapter{Présentation de l’entreprise}
Décrire l’entreprise d’accueil, ses activités principales et son rôle dans le projet.  

% ----------------------
% PROJET
% ----------------------
\chapter{Présentation générale du projet eco-pilote}
\section{Contexte et objectifs}
\section{Utilisateurs cibles}
\section{Fonctionnalités principales}

% ----------------------
% CAHIER DES CHARGES
% ----------------------
\chapter{Cahier des charges}
\section{Besoins fonctionnels}
\section{Besoins techniques}
\section{Contraintes}

% ----------------------
% TECHNOLOGIES
% ----------------------
\chapter{Étude et choix technologiques}
Présentation du choix de la stack \textbf{MERN (MongoDB, Express, React, Node.js)} et justification.  

% ----------------------
% CONCEPTION
% ----------------------
\chapter{Conception du projet}
\section{Architecture de l’application}
Inclure un schéma de l’architecture MERN.  

\section{Diagrammes UML}

\begin{figure}[H]
    \centering
    \includegraphics[width=0.5\linewidth]{image.png}
    \caption{Enter Caption}
    \label{fig:placeholder}
\end{figure}

\section{Modèle de la base de données}

% ----------------------
% IMPLÉMENTATION
% ----------------------
\chapter{Implémentation}
\section{Backend (Node.js + Express)}
\section{Frontend (React)}
\section{Base de données (MongoDB)}
\section{Sécurité et authentification (JWT, hashage)}
\section{Tests et validation}

% ----------------------
% DÉPLOIEMENT
% ----------------------
\chapter{Déploiement}
\section{Environnement de développement}
\section{Environnement de production}
\section{Gestion de versions (Git/GitHub)}
\section{CI/CD (si applicable)}

% ----------------------
% RÉSULTATS
% ----------------------
\chapter{Résultats obtenus}
Inclure captures d’écran de l’application (ex. page de connexion, tableau de bord).  

% ----------------------
% DIFFICULTÉS ET SOLUTIONS
% ----------------------
\chapter{Difficultés rencontrées et solutions}
Lister les problèmes techniques et organisationnels rencontrés ainsi que leurs solutions.  

% ----------------------
% PERSPECTIVES
% ----------------------
\chapter{Perspectives d’amélioration}
Améliorations techniques et fonctionnelles possibles.  

% ----------------------
% CONCLUSION
% ----------------------
\chapter{Conclusion}
Bilan global du stage, apports professionnels et personnels.  

% ----------------------
% ANNEXES
% ----------------------
\appendix
\chapter{Annexes}
\section{Extraits de code}
\section{Documentation API}
\section{Diagrammes complets}

% ----------------------
% BIBLIOGRAPHIE
% ----------------------
\begin{thebibliography}{9}
\bibitem{mern} Documentation officielle MERN : \url{https://www.mongodb.com/mern-stack}
\bibitem{react} Documentation React : \url{https://react.dev}
\bibitem{node} Documentation Node.js : \url{https://nodejs.org}
\end{thebibliography}

\end{document}


\section{Introduction}


\end{document}
